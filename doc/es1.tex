\documentstyle[11pt]{article}
\pagestyle{plain}
\textheight 9.4in
\addtolength{\oddsidemargin}{-0.7in}
\addtolength{\textwidth}{1.2in}
\topmargin -0.3in
\headheight 0.1in
\headsep 0.1in
\begin{document}

%Cover page
\begin{center}
\Huge
\fbox{\fbox{\rule[-.65in]{0mm}{1.in} {\bf \hspace{2in} ES1, XES1 \hspace{1.8in} }}}

\vspace{-.4in}

\Large
{\bf ELECTROSTATIC 1 DIMENSIONAL CODE}

\vspace{0.3in}
REFERENCE MANUAL \\
ES1, XES1 version 4.1

\begin{figure}[h]
\vspace{4.9in}
\special{psfile=2stream.eps voffset=0 hoffset=0 angle=0 hscale=70 vscale=70}
\end{figure}

\normalsize
\copyright Copyright 1987-1994 Regents of University of California \\
John P. Verboncoeur\\
Vahid Vahedi\\
c/o Prof. C. K. Birdsall\\
Plasma Theory and Simulation Group\\
Electronics Research Laboratory, Cory Hall\\
University of California\\
Berkeley, CA 94720

\end{center}

%Table of contents and the rest of the doc
\newpage
\tableofcontents

\newpage
\begin{section}
{\bf INTRODUCTION}

ES1 is an ElectroStatic 1 dimensional code for the MS-DOS operating system.
The code simulates a periodic plasma whose characteristics, including particles
and electrostatic fields, are specified by the user at run time using an input
file.  The simulation proceeds in real-time, with the user viewing output as
the code runs in the form of various diagnostics which change with each time
step (animation).
\vspace{.2in}

\noindent
The MS-DOS version of ES1 is compiled with the Microsoft C 5.1 Optimizing
Compiler.  This version of ES1 is almost faithful to the Cray FORTRAN version
originated by A. B. Langdon, with the exception of the form of input and
output.  The PC version provides interactive animation whereas the FORTRAN
version produces output files requiring post processing.

\noindent
In addition, an X-Windows version is also available for Unix workstations
implementing the MIT X11R4 standard.  The X-Windows versions runs in an X11
environment called XGrafix, with complete source code included.  To compile the
X-Windows version, the standard MIT X11 libraries are required.  The physics
code of the X-Windows version is identical to that of the DOS version.
\begin{subsection}
{\bf Scope}
   This document describes the ES1 program running on the IBM PC and
   compatibles, as well as the X-Windows version.  Program installation,
   operation, and modification are discussed.  In addition, the library of
   input files accompanying ES1 is described, and the guidelines to generate
   new input files are provided.
\vspace{.2in}

\noindent
   This manual makes no attempt to explain the physics and computational issues
   of particle simulation.  To learn more about particle simulation in general
   and specifically the physics behind ES1, refer to {\em Plasma Physics Via
   Computer Simulation} by C. K. Birdsall and A. B. Langdon (McGraw Hill 1985,
   Adam-Hilger and IOP 1991).  Birdsall and Langdon devote about one third of their text
   to a detailed discussion of ES1 and its applications.
\vspace{.2in}

\noindent
   ES1 is intended as a companion package to that text, although the text is
   not required to use the program.  In either case, some familiarity with
   plasma physics is required to understand the results of the simulations and
   generate new simulations.  Knowledge of numerical analysis and/or particle
   simulation is useful for modification of the code and understanding of
   numerical errors which can occur in any computer simulation.
\end{subsection}

\begin{subsection}
{\bf Objectives}

   The objectives of this package include distribution of the ES1 code in a
   form which makes it accessible to large and small research facilities,
   universities, and individuals interested in plasma simulation.  The IBM PC
   environment provides a vehicle for distribution since nearly every
   researcher and student has access to this type of machine.  The trend in
   recent years has been one of decreasing cost and increasing performance and
   capability.
\end{subsection}

\begin{subsection}
{\bf History}

   ES1 was originally written by A. B. Langdon (about 1972) for a course at University of
   California, Berkeley taught by C. K. Birdsall.  The code is both an
   educational tool and the prototype for many computer simulations.  ES1
   illustrates many fundamental concepts of plasma simulation, including
   electrostatic movers, various techniques of particle weighting, noise
   reduction (smoothing), Maxwellian particle loading, etc.  The original
   FORTRAN code by Langdon has been updated many times; the current version is
   available to users of the National Energy Research Supercomputer Center at
   Lawrence Livermore National Laboratory from Langdon.  We have also included
   the source code with this distribution (refer to Section 2.1 Disk Contents
   for the filename).
\vspace{.2in}

\noindent
   ES1 was translated to C in 1985 by T. Lasinski, and ran on IBM XT compatible
   computers with CGA resolution video.  Subsequent modifications are detailed
   in Section 1.4 Enhancements.
\end{subsection}

\begin{subsection}
{\bf Enhancements}
   This section summarizes the enhancements made to the PC version of ES1 since
   the original version written in Desmet C by T. Lasinski.  The enhancements
   are presented in the order they occurred, referenced by version number.
   Note that the original version has been given version number 1.0 although it
   had no explicit version number.  There have been a large number of upgrades
   distributed since we have attempted to respond quickly to requests for
   specific fixes and improvements.

\begin{subsubsection}
{\bf Version 2.0}

      This upgrade includes a major conversion of the source code from Desmet
      CWare compiler to Microsoft C 5.
\vspace{.2in}

\noindent
      The graphics have been improved by a factor of nearly 4 in resolution
      ($320\times200 \rightarrow 640\times350$ pixels on the EGA) 
      and by a factor of 4 
      in color (4 colors $\rightarrow$ 16 colors). In addition, the
      program runs about twice as fast
      as older versions, and has the capability to use up to 16000 particles,
      8192 grid spaces, 8 species, and 1024 time steps, subject to available
      memory.
\vspace{.2in}

\noindent
      The diagnostics have been improved to include electric field, electric
      potential, kinetic energy history, and total energy history.  Also the
      animation mode we call 'noerase' (see discussion of noerase below) was
      added.
\end{subsubsection}

\begin{subsubsection}
{\bf Version 2.1}

      The major improvement included in this version is the dynamic allocation
      of all major arrays.  This decreased the size of the executable file from
      350K bytes to about 58KB.  This fix also enables the program to run small
      simulations on machines equipped with 256K memory, while handling larger
      simulations on machines with more memory.  Speed is not affected much
      because the program is loaded from disk faster, but the program must
      dynamically request memory at run time and initialize the arrays as
      required.
\end{subsubsection}

\begin{subsubsection}
{\bf Version 2.101}

      The animation screen was enlarged to provide more resolution, and support
      for IBM CGA in high resolution mode was added.
\end{subsubsection}

\begin{subsubsection}
{\bf Version 2.102}
      An error was discovered in the Maxwellian thermal velocity loaders (used
      when a non-zero VT1 or VT2 value occurs in the input file).  The loaders
      now are normalized properly and give the correct spread of velocities.
\end{subsubsection}

\begin{subsubsection}
{\bf Version 2.103}

      An error was discovered in the way ES1 rotates velocities in Vx-Vy space
      for magnetized plasmas.  Previously, the rotation was correlated to the
      spatial position (i.e. the particles were not loaded randomly in velocity
      space).  The particles now are loaded correctly in velocity space.
\end{subsubsection}

\begin{subsubsection}
{\bf Version 2.104}

      General optimizations were performed to streamline much of the code which
      is called multiple times.  The optimizations include use of 8086 register
      storage for critical variables.  The overall improvement in speed is
      about 10\% over previous versions.
\end{subsubsection}

\begin{subsubsection}
{\bf Version 2.105}

      This version adds a perpendicular k-space component to (artificially) in order to
      simulate two dimensional charge disks.  Note that when a value is not
      specified or $a \rightarrow \infty$, ES1 assumes the standard 1d case of infinite charge
      sheets.  Charge disks become infinite charge sheets as $a \gg l$.
      Also, the
      charge disks are still really one dimensional as far as the axis of
      symmetry is concerned - collisions are not avoided in the case of charge
      disks in a magnetic field because ES1 is still 1d2v.  Thus we do not
      track the y or radial position of particles.
\vspace{.2in}

\noindent
      Since ES1 is a one-dimensional simulation, it normally uses infinite
      charge sheets instead of three-dimensional particles.  In PC ES1, we have
      implemented an optional second dimension to simulate charge disks.   The
      one-dimensional Poisson equation is given by
\begin{center}
	{$-\nabla^{2}\phi(k) = {\rm K}^{2}\phi(k) =
	{\rho(k) \over \epsilon_{0}}$} , 
\end{center}
\noindent
	where
\begin{center}
	{${\rm K}^{2} = k^{2} \left( {\sin{k \Delta x \over 2} \over 
	{k \Delta x \over 2}} \right) ^{2}$} .
\end{center}
\noindent
	To simulate charge disks, K is augmented by a perpendicular component:
\begin{center}
	{${\rm K}^{2} = k^{2} \left( {\sin{k \Delta x \over 2} \over 
	{k \Delta x \over 2}} \right ) ^{2} + \left( 2.405 \over a \right)^{2}$		} .
\end{center}
\noindent
      Here, $a$ is the finite radius of the charge disk and 2.405 is the first
      zero of the Bessel function ${\rm J}_{0}$.  If $a$ is not specified, infinity is
      assumed and the charge disks become infinite charge sheets.  Note that
      the while the finite radius of a charge disk affects the potential,
      particles are still constrained to motion in one dimension (although they
      can have a perpendicular velocity).

\noindent
To simulate a background of Boltzmann electrons, using positive ions as the only
moving particles, $K^2$ is augmented by:
\begin{center}
	{${\rm K}^{2} = k^{2} \left( {\sin{k \Delta x \over 2} \over 
	{k \Delta x \over 2}} \right ) ^{2} + k_D^2$ } where
	$k^2_D = \left( \frac{ n_{e0}e^2}{m\epsilon_0}\right)/ \left( \frac{kT_e}{m}\right)$
\end{center}
\noindent
The code uses $l/a = l/2.405{\lambda_D}=0.4158l/\lambda_D$.
\end{subsubsection}

\begin{subsubsection}
{\bf Version 2.106}

      In this version we add energy histories for the Fourier-decomposed field
      energy modes 1, 2, and 3.  These diagnostics can be viewed by pressing
      the number corresponding to the desired mode energy.  The display is
      similar to the field energy history.
\end{subsubsection}

\begin{subsubsection}
{\bf Version 2.107}

      In this version, the display paging scheme (the way animation is
      accomplished in ES1) was enhanced.  The NoErase particle traces are
      retained even when calling up a diagnostic plot, and screen items are
      written more efficiently to improve speed.
\end{subsubsection}

\begin{subsubsection}
{\bf Version 2.108}

      With this version, the file extension of .INP is assumed for input files
      by default.  Thus to run the landau damping input file, LANDAU.INP, one
      could type {\sc ES1 LANDAU.INP} or equivalently {\sc ES1 LANDAU}.  Note 
      that
      input files with other extensions can still be run.  For
      example, {\sc ES1 LANDAU.NEW} would run the input file LANDAU.NEW,
      while {\sc ES1 TEMP} would
      first look for the input file TEMP with no extension, and if TEMP does
      not exist then ES1 looks for TEMP.INP.
\end{subsubsection}

\begin{subsubsection}
{\bf Version 2.109}

      The input file format was revised to an enable arbitrary number of
      comment lines between text.  The text lines need not be preceded by any
      specific character - ES1 assumes a line is a text line if it does not
      contain the next expected data.  The lines of numbers are still of the
      same format; there must be no interceding text between numbers other than
      spaces or tabs.
\end{subsubsection}

\begin{subsubsection}
{\bf Version 2.11}

      Finally we have semi-log plots for the all energy histories.  Now
      exponential growth rates should be easier to see as they become straight
      lines on the semi-log plots.  The energies are now stored in log10 form,
      requiring an evaluation of a log for each energy each timestep rather
      than {\em nt} evaluations when a time history is requested.
\end{subsubsection}

\begin{subsubsection}
{\bf Version 3.0}

      Version 3.0 is a major upgrade providing ES1 with an improved user
      interface and output capabilities.  The core of the user interface,
      designed and written by John Verboncoeur and Vahid Vahedi, handles the
      programming issues of the simulation.  These include keyboard handling,
      screen graphics, and printer output for PostScript and IBM Graphics
      printers (including the Epson FX compatible family of printers).
\vspace{.2in}

\noindent
      The code is separated into a physics application and the windowing core.
      New physics and diagnostics can be added without altering the windowing
      code, with the only restriction that any new diagnostic must be a linear,
      semi-log, or scatter plot.  A text plot is currently under consideration
      which would display parameters from the input file during the simulation.
\vspace{.2in}

\noindent
      Using the windowing core, all diagnostics are updated dynamically in
      time.  The core can also update in individual timesteps, pausing for a
      keystroke before continuing the simulation.
\vspace{.2in}

\noindent
      The speed is reduced on the order of 50\% compared to 
      previous versions
      with many plots displayed.  The code is faster than previous versions if
      it is run with no diagnostics since it no longer processes the diagnostic
      and graphics code.
\vspace{.2in}

\noindent
      For magnetized species, ES1 can now display velocity space (${\rm v_{x}}$ 
	versus
      ${\rm v_{y}}$).  This diagnostic is useful for loading ring distributions or
      observing transverse heating.  The diagnostic only appears when at least
      one species is magnetized (i.e. non-zero cyclotron frequency).
\vspace{.2in}

\noindent
      ES1 no longer has a time step limitation; it can run indefinitely.  All
      time histories are combed periodically such that there are never more
      than 1024 (subject to change) values stored.  Note that after long runs
      this can result in a loss of high frequency resolution on the history
      plots.  This has no effect on the physics of the simulation.
\vspace{.2in}

\noindent
      ES1 can now generate hardcopy output on an IBM Graphics Printer or
      compatible, including the Epson FX series of dot matrix printers.  In
      addition ES1 can generate a PostScript file which can be sent to any
      PostScript printer or imported into a PostScript document.
\end{subsubsection}

\begin{subsubsection}
{\bf Version 3.1}

      An error was discovered in the species initialization function, init(),
      in calculating the charge of each species.  The charge per unit
      area, $Q$,
      for each species is calculated from the other input parameters,
      $\omega_{p}$
      (the plasma frequency), $N$ (the number of particles),
      $qm$ (the charge to mass ratio), $l$ (the system length), and
      $epsi$($1/{\rm \varepsilon}_{0}$) to be:
\begin{center}
      $Q = l\omega_{p}^{2}/(N*qm*epsi)$ .
\end{center}
\noindent
      Previously the term {\em epsi} was not there.  Typically {\em epsi} 
      is set to one for
      normalized parameters, so its absence was not noticed.  However, for
      running with laboratory parameters, {\em epsi} must be set to
      $1 / \varepsilon_{0}$, where ${\rm \varepsilon}_{0}$ is the vacuum
      dielectric constant (or some multiple of it for other materials).
\end{subsubsection}

\begin{subsubsection}
{\bf Version 4.0 (XES1)}

      XES1 is the X-Windows version of ES1 running under X11.  This version
      makes extensive use of a mouse for selecting menus, moving windows,
      resizing, etc.  In addition, XES1 can be run on a remote host while the
      output displays on a local graphics workstation.  XES1 uses XGrafix for
      the graphics display, which requires X11 libraries to compile.
\end{subsubsection}
\begin{subsubsection}
{\bf Version 4.1 (ES1,XES1)}
\noindent

Higher order weightings were added to the move and accel subroutines.  The
meaning of the iw flag was changed, and a new flag controlling weightings was added.
Velocity distribution diagnostics were added; 3 new species parameters and one
new global parameter were needed to allow the use of these diagnostics.

\end{subsubsection}

\end{subsection}
\end{section}

\newpage
\begin{section}
{\bf INSTALLATION}

This section describes the contents of the ES1 distribution disks and the
installation procedure for hard disk and floppy disk systems.
\begin{subsection}
{\bf Disk Contents}

   ES1 is distributed on a single 1.44M diskette.  This disk contains the files
   and directories for the IBM compatible PC version as well as the X Windows
   version.
\vspace{.2in}

\noindent
   The ES1 directory contains the required files for the PC version.  These
   are:

\begin{tabbing}
        README.TXTaaaaa\= ES1 executable file.  This file must be in the current directory \=	\kill
        ES1.EXE      \> ES1 executable file.  This file must be in the current directory \>	\\
	 	     \> or on the current path to run ES1.  See your DOS manual for 	\>	\\
		     \> information on paths and directories.  					\\
												\\
        INSTALL.BAT  \> Installation program for the PC version to copy all the relevant\> 	\\
		     \> ES1 files into appropriate directories.  				\\
												\\
        README.TXT   \> Text file containing information for programmers wishing to 	\> 	\\
		     \> modify ES1.  When the information in this manual conflicts 	\>	\\
		     \> with the README.TXT file, assume the file is correct.  This 	\> 	\\
		     \> file may not be present if this manual is up to date with the 	\> 	\\
                     \> version of the code distributed.  					\\
												\\
        *.C          \> All files with the C extension are the C language source files 	\> 	\\
		     \> for ES1. 								\\
												\\
        *.H          \> All files with the H extension are the C language header files 	\>	\\
		     \> for ES1. 								\\
												\\
        ES1.MAK      \> The MAKE utility file for automatically performing conditional 	\> 	\\
		     \> compilation/linking of only those files which have been 	\> 	\\
		     \> changed.  Requires the Microsoft MAKE utility provided with 	\> 	\\
		     \> many recent compilers (including C 5, Quick C 1, FORTRAN 	\> 	\\
		     \> 4, and MASM 5). 							\\
												\\
        MK.BAT       \>   Batch file for automating compilation.			\>
\end{tabbing}
\noindent
   The WIN directory contains the object files for displaying graphics in the
   PC version.  These are:
\begin{tabbing}
        WINGRAPH.OBJaaaaa\= Object code (Microsoft format) for thewindowing core used	\= \kill
        WINGRAPH.OBJ \> Object code (Microsoft format) for thewindowing core used	\> \\
        WINTOOLS.OBJ \> by ES1.  Required only if the application will be modified 	\> \\
		     \> and recompiled.  These files are placed in the \verb1\1WIN 	\> \\ 
		     \> directory.							\> \\
											   \\
        WINGRAPH.H   \> header file for windowing core features. Required only if 	\> \\ 
		     \> the application will be modified and recompiled.  This file is  \> \\
		     \> placed in the \verb1\1WIN directory.				\>
\end{tabbing}

\newpage
\noindent
   The INP directory contains the input files for the PC version.  These are:
\begin{tabbing}
        *.inpaaaaa\=  All files with the .inp extension are input files.
For detailed information \= \kill 
        *.inp  \>  All files with the .inp extension are input files. For detailed information \> \\
	       \> on each input file refer to Section 5.2 Input File Library.  A directory called \> \\
		\> inp can set up under the es1 directory (for the PC version) or the xes1 \> \\
		\> directory (for the X Windows version) to include all the input files. \> 
\end{tabbing}
\noindent
   The FES1 directory contains a tar file, ES1.TAR.Z containing the source files
   for the FORTRAN version of ES1.  The FORTRAN version uses NCAR graphics and
   has its own documentation.
\vspace{.2in}

\noindent
   The DOC directory contains this manual in two formats:
\begin{tabbing}
        ES1.TXTaaaaaaaaa\=       The PostScript version of this document. \= \kill
        ES1.PS  \>       The PostScript version of this document. \> \\
								\\
        ES1.TEX \>       The LaTeX version of this document. \> \\
								\\
        2STREAM.PS \>       A PostScript figure for this document. \> 
\end{tabbing}

\noindent
   The XES1 directory contains a README file and a tar file XES1.TAR.Z which
   contains the files required for XES1.  The README file contains
   directions for installing XES1 on a Unix platform.  Once the compressed tar
   file is installed, two directories are created: {\em xes1} 
   and {\em xgrafix}.  The {\em xes1} directory contains:

\begin{tabbing}
	READMEaaaaaaaaa\=      All files with the .c extension are the C language source files for XES1. \= \kill
        *.c    \>       All files with the .c extension are the C language source files for XES1. \> \\
	       \>      These files should be placed in the xes1 directory. \> \\ 
												     \\
        *.h   \>       All files with the .h extension are the C language header files for XES1.  \> \\
		\>    These files should be placed in the xes1 directory. \> \\ 
										\\
        makefile  \>     The make file for automatically performing conditional compila- \> \\
		  \> tion/linking of only those files which have been changed.  This file \> \\
		  \> should be placed in the xes1 directory. \> \\
												\\
        README.license    \>     Contains the license agreement for XES1. \> \\
										\\
        README.upgrade   \>     Contains information on how to obtain the latest versions of ES1. \> \\
										\\
        {\em /inp}\>      Subdirectory containing the input files for XES1, *.inp.  These are the \> \\
		  \>   same input files used by the PC version. \> \\
										\\
        {\em /doc}\>   Subdirectory containing the documentation files es1.tex (LaTeX version), \> \\
		  \>   2stream.ps, and es1.ps (PostScript version). \> 
										\\
\end{tabbing}
\noindent
   The {\em xgrafix} directory contains the files for XGrafix including the
   following:

\begin{tabbing}	
	xgrafix.icoaaaaa\= The source file for the XGrafix graphics display library.  This file \= \kill
        *.c   \> The source files for the XGrafix graphics display library.  This file \> \\
		    \> should be placed in the xgrafix directory. \> \\
									\\

        xgrafix.h   \>   Header file for XGrafix.  This file should be
                       placed in the xgrafix \> \\
		    \> directory. \> \\	
										\\

        xgrafix.ico \>   The XGrafix icon (bitmap).  This file should be placed in the xgrafix directory. \> \\	
										\\

        xgrafix.str \>   Another header file containing string definitions for XGrafix.  This \> \\
		    \> file should be placed in the xgrafix directory. \> \\
									\\
        Imakefile    \> The Imakefile file for XGrafix.  This file should be placed in the xgrafix \> \\
		     \> directory.  If the X libraries are installed properly, run {\em xmkmf} to generate \> \\
		     \> a Makefile, and then type {\em make} to create the XGrafix libraries (libXGC.a \> \\
		     \> \& libXGF.a). In this case you will not need to use Makefile.xgrafix. \> \\
								\\
        Makefile.xgrafix    \>  The imakefile file for XGrafix.  This file should be placed in the xgrafix \> \\
		    \> directory. If you can not find {\em xmkmf} on your system, modify this file to \> \\
		    \> compile and create the XGrafix libraries. \> 
								\\
\end{tabbing}
\end{subsection}

\begin{subsection}
{\bf Setup and Installation Procedure (PC version)}

   The setup procedure described in this section assumes the user is familiar
   with the DOS COPY and MD (Make Directory) commands.  The setup of ES1 is
   similar for single and dual floppy disk drive systems as well as hard disk
   systems.  ES1 is capable of detecting the hardware configuration of the
   system, including the video adapter, numeric coprocessor, etc., and
   automatically adjusts to use the hardware to best advantage, so there is no
   setup procedure required for these items.
\vspace{.2in}

\noindent
   If you will be using ES1 on a floppy-based system, backup the ES1
   distribution disk using the DOS DISKCOPY program.  Do not remove the write
   protect tab on the original ES1 diskette to ensure that you retain a copy of
   the original, unmodified executable code, source code, and input files.
   This completes the installation procedure for floppy disk systems.
\vspace{.2in}

\noindent
   For automated installation to any type of disk, make the floppy disk
   containing the ES1 files the current disk by entering {\sc A:} at the DOS
   prompt.  Enter {\sc INSTALL C:} to copy the files to drive C: 
   (replace 'C' by
   the appropriate drive letter if you have more than one logical hard disk
   drive).  This program will create a directory \verb1\1ES1 and copy 
   the ES1 files to
   it, and will also create a directory \verb1\1WIN and copy the object
   and header
   files for the windowing core.  Note that the contents of the \verb1\1WIN
   directory are only required if the code will be modified.
\vspace{.2in}

\noindent
   To use ES1 in these configurations, you must first make the ES1 directory
   the current directory by typing {\sc CD ES1}.  Then follow the 
   directions below for program operation.
\end{subsection}

\begin{subsection}
{\bf Setup and Installation Procedure (X Windows version)}

   The installation procedure for the X Windows version must be done manually.
   Only the contents of the XES1 directory are needed for this version.  Place
   the contents of the XES1 directory in the home directory on the Unix
   machine.  Follow the directions shown in the file README to extract ES1.TAR
   and install files in the appropriate directories.
\end{subsection}
\end{section}

\newpage
\begin{section}
{\bf MS-DOS PROGRAM OPERATION}

\begin{subsection}
{\bf Syntax}

   ES1 INP\verb1\1filename[.inp],
\vspace{.2in}

\noindent
   where $<{\rm filename.inp}>$ is the name of the input file.  Although we have used
   *.INP for the input files in the library, the .INP extension is not
   required.  If no filename is provided on the command line, ES1 looks for a
   file named ES1DATA (to maintain compatibility with previous versions).  If
   the input files are not in the same directory or are located in a
   sub-directory, the path must also be specified.  For instance, the syntax
   for starting ES1 with the input file 2stream.inp which is in a sub-directory
   of es1 called INP is:
\vspace{.2in}

\noindent
   ES1 INP\verb1\12STREAM[.INP]
\vspace{.2in}

\noindent
   The input file is required since ES1 determines the parameters of the
   simulation at run time.  The input parameters are displayed and the computer
   pauses for a keystroke.
\end{subsection}

\begin{subsection}
{\bf Run, Stop, and stEp}

   ES1 is a real-time simulation; you can view the results of the simulation as
   they occur.  When the simulation is paused (i.e. time is not being
   incremented), there is a menu option {\bf Run} available.  When
   {\bf Run} has been
   selected, it is replaced on the menu by {\bf Stop}.  Thus to 
   initiate or continue
   the simulation, select {\bf Run}.  To pause the simulation, 
   select {\bf Stop}.  Note
   that ES1 is not limited in the number of timesteps since time histories are
   combed periodically.  The timestep counter may become negative after 32k
   timesteps, but this affects only the displayed number; internally the
   physics are still consistent, and the timestep will continue to increment,
   eventually becoming positive again.
\vspace{.2in}

\noindent
   The {\bf stEp} option causes ES1 to take a single timestep and 
   pause.  This option
   is available at all times.  {\bf stEp} is useful for viewing 
   discrete temporal
   changes and studying a rapidly changing simulation in detail before the
   phenomena of interest (an instability, for example) is completed.  When the
   simulation is running (after selecting the {\bf Run} option), 
   {\bf stEp} has the affect
   of stopping the simulation and then taking a single step, leaving the
   simulation paused ({\bf Run} appears on the menu and can be selected).
\end{subsection}

\begin{subsection}
{\bf Tools}

   This menu item provides the tools for modifying the diagnostics windows.
   The services provided by {\bf Tools} include moving windows to 
   a new location,
   changing the size of the windows, changing the plotting limits for the
   windows (including an automatic rescale), and turning on noerase for a
   window.  All features provide interactive directions while in use.

\begin{subsubsection}
{\bf Move}

      The {\bf Move} feature enables windows to be moved to a new 
      location on the
      display.  After selecting {\bf Move}, the list of diagnostics 
      is presented.  To
      select a window, move the highlight bar to the appropriate item and press
      \fbox{Enter} or press the first letter of the desired item.  If the
      window is currently on the screen, the outline is highlighted while the
      user drags the window to the new location using the cursor keys.  If the
      window is not currently active, its outline appears and it is made active
      once the {\bf Move} operation is completed.  The cursor keys move the outline
      in 1 pixel increments, while the cursor keys in conjunction with shift
      move the outline in 8 pixel increments.  When the diagnostic has been
      moved to the desired location, press \fbox{Enter} to complete the {\bf Move}
      operation.
\end{subsubsection}

\begin{subsubsection}
{\bf Resize}

      The {\bf Resize} feature provides the capability to alter the size of a
      displayed window.  To change the size, select {\bf Tools, Resize}.  The list of
      available diagnostics is displayed.  After selecting a diagnostic, ES1
      prompts for a window edge to resize.  Select the edge using the cursor
      keys (or shift-cursor keys).  The window outline is highlighted, and the
      selected edge can be resized in either direction.  When the edge has been
      moved to the desired size, press \fbox {Enter}.  At this point, the user
      is prompted to select another side to resize.  Continue to select edges
      to resize until the desired size is obtained.  Pressing \fbox{Enter} at
      the select edge prompt completes the operation.
\end{subsubsection}

\begin{subsubsection}
{\bf rEscale}

      To change the displayed limits of a plot select {\bf rEscale}.  A list of
      available diagnostics is displayed.  This function only works on active
      diagnostics.  After selecting an active diagnostic, a dialog box
      containing the scaling information for the window is displayed.  The
      items displayed include the upper and lower limit of the x- and y- axis,
      the Autorescale toggle for each axis, and the OK and CANCEL buttons.  Use
      the cursor keys to move the highlight to the desired item.  To change a
      number, enter the new number and press \fbox{Enter} or 
      \fbox{$\downarrow$}.  To
      toggle Autorescale, press any key.  To apply the changes made, select the
      OK button and press \fbox{Enter}.  To cancel changes, select CANCEL and
      press \fbox{Enter} or press \fbox{Esc}.
\vspace{.2in}

\noindent
      The x- and y- axis labels do not affect the simulation; only the portion
      of the plot which can be seen is affected.  Autorescale causes the
      windowing core to update the axis labels each time step based on the
      maximum and minimum values.  For linear plots, the maximum or minimum is
      found each time step.  For semi-log plots, autorescale will only
      increment the labels in decades.  To ensure that the plot does not span
      too many decades due to a wide variation on a log scale (such as a
      quantity which goes nearly to 0), autorescale selects at most 6 decades
      to plot starting from the maximum value.
\vspace{.2in}

\noindent
      Note that autorescale now works correctly on plots with a multiple items;
      i.e., a phase space plot with several species can now be autorescaled.
\end{subsubsection}

\begin{subsubsection}
{\bf Noerase}

      The {\bf Noerase} item provides the capability to leave the plot at the
      previous timestep on the screen when drawing the new plot.  This is often
      useful for watching the trajectories of particles in phase space to
      determine trapping regions or holes in phase space, or to observe the
      transient behavior of a snapshot for a brief time.
\vspace{.2in}

\noindent
      When {\bf Noerase} is selected, the list of available diagnostics is
      displayed.  The operation is completed by selecting a diagnostic window
      to apply the noerase feature to.
\vspace{.2in}

\noindent
      Note that there are times when the noerase plot cannot be saved.  This
      includes cases when another window overlaps the window with noerase on,
      as well as resizing, moving or rescaling the diagnostic.
\end{subsubsection}

\begin{subsubsection}
{\bf Cross-hair}

      The {\bf Cross-hair} tool provides the user with a pointer to obtain the
      coordinates of any point in any window (in coordinate system of the
      window e.g. semilog, etc.).  The coordinates are displayed in the upper
      left corner of the screen.
\vspace{.2in}

\noindent
      When {\bf Cross-hair} is selected, the simulation is paused.  
      Pressing cursor
      keys will move the cross-hair in vertical and horizontal directions.  To
      exit and resume simulation press \fbox{Esc} or \fbox{Enter}.
\end{subsubsection}
\end{subsection}

\begin{subsection}
{\bf Diagnostics}

   ES1 provides diagnostics for many simulation parameters of interest.
   Diagnostics can be viewed at any time and in any arrangement desired.  The
   diagnostic windows can be resized, moved, rescaled, and closed.  Note that
   the program speed is affected slightly by the number of diagnostics
   displayed since each diagnostic requires some additional processing each
   timestep when it is open.
\vspace{.2in}

\noindent
   Diagnostics are opened or closed in the same manner; the 
   {\bf Diagnostic, $<$menu item$>$} sequence is a toggle which opens 
   the diagnostic if previously
   inactive.  If the diagnostic is currently active, it will 
   have a marker next
   to its name in the list of diagnostics.  Selecting an active diagnostic
   makes it inactive.

\begin{subsubsection}
{\bf VX Phase Space}

      The VX Phase Space diagnostic contains the menu bar as well as the plot
      of velocity versus position for all particles of all species.  Particles
      of different species are assigned different colors to distinguish them.
      Often different species are used for particles of different
      characteristics such as charge, mass, initial velocity, etc.  Note that
      since ES1 is a periodic code, particles which exit one side of the Phase
      Space plot will return through the other side with the same velocity.
      Although ES1 processes velocities and positions in normalized form, the
      plots are denormalized to reflect values in the same units as the input
      parameters in the input file.
\vspace{.2in}

\noindent
      VX Phase Space is displayed by selecting 
      {\bf Diagnostics, VX Phase Space} from
      the menu.
\end{subsubsection}

\begin{subsubsection}
{\bf Electric Field}

      The Electric Field diagnostic displays the value of the electric field at
      the grid points.  Since the electric field is the spatial integral of the
      density, it is smoother.  ES1 uses the electric field to determine the
      electrostatic force on particles in the mover.  The electric field is
      also periodic.
\vspace{.2in}

\noindent
      Electric Field is displayed by selecting {\bf Diagnostics, Electric Field}
      from the menu.
\end{subsubsection}

\begin{subsubsection}
{\bf Potential}

      The Potential diagnostic displays the value of the potential at the grid
      points.  Since the potential is the spatial double integral of the
      density, it contains even less noise than the electric field.  The
      potential is also periodic.
\vspace{.2in}

\noindent
      Potential is displayed by selecting {\bf Diagnostics, Potential}
      from the menu.
\end{subsubsection}

\begin{subsubsection}
{\bf Density}

      The Density diagnostic displays the space charge density of all particles
      as it is weighted to the grid.  Note that the density is periodic; i.e.,
      ${\rm \rho}[0] = {\rm \rho}[NG]$.
      The density may be noisy due to the small
      number of particles being weighted to a finite grid.
\vspace{.2in}

\noindent
      Density is displayed by selecting {\bf Diagnostics, Density} from the menu.
\end{subsubsection}

\begin{subsubsection}
{\bf Kinetic Energy}

      The Kinetic Energy diagnostic displays the time history of the total
      kinetic energy for all species.  This history is a semi-log plot.
\vspace{.2in}

\noindent
      To display the Kinetic Energy history, 
      select {\bf Diagnostics, Kinetic Energy} from the menu bar.
\end{subsubsection}

\begin{subsubsection}
{\bf Total Energy}

      The Total Energy diagnostic displays the time history of the total energy
      of all species; i.e. total energy = kinetic + field energies.  Note that
      this history is a semi-log plot, calculated from
\begin{center}
      $\log(TE_{i}) = \log(KE_{i} + ESE_{i})$ .
\end{center}
\noindent
      The total energy of a given simulation should be conserved; numerical
      error is indicated when the total energy changes.  In simulations where
      the total energy changes by more than a few percent, there is serious
      error in the solution possibly resulting from a timestep exceeding the
      limit:
\begin{center}
      $\omega \Delta t \leq 2$ .
\end{center}
\noindent
      Here $\omega$ is the highest frequency in the simulation.  Exceeding
      the limit results in exponential growth and damping in the leapfrog
      movers.  See Birdsall and Langdon for a detailed analysis.
\vspace{.2in}

\noindent
      To display the Total Energy diagnostic, select 
	{\bf Diagnostics, Total Energy} from the menu.
\end{subsubsection}

\begin{subsubsection}
{\bf Field Energy}

      The Field Energy diagnostic displays the time history of the
      electrostatic field energy.  Note that this is a semi-log plot, enabling
      the viewer to determine exponential growth/damping rates of energy more
      easily (e.g., the growth rate of an instability).
\vspace{.2in}

\noindent
      Field Energy is displayed by selecting 
	{\bf Diagnostics, Field Energy} from the menu.

\vspace{.2in}

\noindent
{\bf 3.4.7.1 Fourier Modes} 		\\

\vspace{.1in}

\noindent
The Fourier modes of the electrostatic energy, Mode 1 ESE, Mode 2 ESE,
etc., display the respective mode of the Fourier spatially decomposed
electrostatic energy.  These history plots are semi-log, emphasizing
exponential growth rates.  Modes up to NG/2 are available from the
physics of the simulation; the displayable modes are set by MMAX in
the input file.
\vspace{.2in}

\noindent
         To display the fundamental mode energy history, select 
	 {\bf Diagnostics, Mode 1 ESE} from the menu.  To display the 
	 second mode energy history, select {\bf Diagnostics, Mode 2 ESE}
  	 from the menu.  To display the n{\em th} mode
         energy, select {\bf Diagnostics, Mode} {\em n} {\bf ESE} from the menu.
\end{subsubsection}

\begin{subsubsection}
{\bf Velocity Space}

      The Velocity Space diagnostic displays a scatter plot of the
${\rm v_{y}}$ versus ${\rm v_{x}}$
      velocity space of magnetized particles only.  Note that ${\rm v_{y}}$ is identically
      zero for unmagnetized species since ES1 is a 1d2v code (1 spatial
      dimension, 2 velocity components).  The scale is the same for both axes.
\vspace{.2in}

\noindent
      Velocity Space is displayed by selecting 
      {\bf Diagnostics, Velocity Space} from
      the menu.
\end{subsubsection}
\end{subsection}

\begin{subsection}
{\bf Print}

   The {\bf Print} menu item provides printing capability for the application.
   Currently supported are two types of printing: a screen dump to a dot matrix
   printer (IBM Graphics printer or compatibles, including the Epson FX
   series), or a PostScript dump to a file for later printing on a PostScript
   laser printer.

\begin{subsubsection}
{\bf PostScript}

      This option generates a high resolution (300 dots per inch on most laser
      printers) vector plot of the currently displayed diagnostics on the
      screen.  The vector plot is stored in an ASCII file in the PostScript
      page description language for later printing and/or inclusion in
      PostScript documentation.  Note that the file can become quite large (up
      to several hundred k bytes) if phase space/velocity space is plotted for
      a large number of particles.  Linear and semilog plots do not add much
      overhead to the file.
\vspace{.2in}

\noindent
      The plot corresponds closely to the screen display with higher
      resolution.  Colors are generally translated into gray shades, with the
      exception of scatter plots, which translate colors into different symbols
      so the various species can be distinguished.
\vspace{.2in}

\noindent
      When the {\bf PostScript} item is selected, a dialog box is displayed.  The
      items contained on the dialog box include the filename to dump into, the
      dump period, and the dump limit.
\vspace{.2in}

\noindent
      The default filename is {\em output.ps}.  To change to a new filename, move the
      highlight bar to the filename field and enter a new filename.  If the
      file already exists from a previous session, it is replaced by the new
      file.  If the file was previously created during the current session, the
      new plot is appended to it.  Note that the file is placed in the current
      directory (normally the \verb1\1ES1 directory).
\vspace{.2in}

\noindent
      The dump period is the number of timesteps which pass between generation
      of plots.  If the dump period is set to 100 at timestep 100, a plot is
      generated for timestep 100, 200, 300, ... until the dump period is
      changed or the maximum number (see dump limit) of plots have been
      generated.  To dump only at the current time regardless of the setting of
      dump limit, use 0 for dump period.
\vspace{.2in}

\noindent
      The dump limit is the maximum number of plots which are to be generated.
      This feature is used in conjunction with dump period to create a file
      containing several plots spaced an equal number of timesteps apart.
\vspace{.2in}

\noindent
      When all desired options have been set, move the highlight bar to OK and
      press \fbox{Enter}.  To exit without plotting to the file and restoring
      old settings, move the highlight bar to CANCEL and press \fbox{Enter}.
\end{subsubsection}

\begin{subsubsection}
{\bf CGM}

      This option generates a high resolution (16-bit integer) vector plot of
      the currently displayed diagnostics on the screen.  The vector plot is
      stored in a binary file conforming to a subset of the ANSI CGM Standard
      Version 1.  The CGM file can be read by many graphics programs for later
      editing, printing, and/or inclusion in documentation.  Note that CGM
      files are more compact than the equivalent PostScript file, but may still
      become large if phase space/velocity space is plotted for a large number
      of particles.  Linear and semilog plots do not add much overhead to the
      file.
\vspace{.2in}

\noindent
      The plot corresponds closely to the screen display with higher
      resolution.  Colors are generally translated into gray shades, with the
      exception of scatter plots, which translate colors into different symbols
      so the various species can be distinguished.
\vspace{.2in}

\noindent
      When the {\bf CGM} item is selected, a dialog box is displayed.  The items
      contained on the dialog box include the filename to dump into, the dump
      period, and the dump limit.
\vspace{.2in}

\noindent
      The default filename is {\em output.cgm}.  To change to a new 
      filename, move
      the highlight bar to the filename field and enter a new filename.  If the
      file already exists from a previous session, it is replaced by the new
      file.  If the file was previously created during the current session, the
      new plot is appended to it.  Note that the file is placed in the current
      directory (normally the \verb1\1ES1 directory).  If multiple 
      files are to be
      generated (dump period $>$ 0, dump limit $>$ 1), they are named in sequence
      by appending a number to the end of the filename.  Some characters at the
      end of long filenames may be removed to fit the plot number.  This is
      necessary since CGM defines only a single page per file.
\vspace{.2in}

\noindent
      The dump period is the number of timesteps which pass between generation
      of plots.  If the dump period is set to 100 at timestep 100, a plot is
      generated for timestep 100, 200, 300, ... until the dump period is
      changed or the maximum number (see dump limit) of plots have been
      generated.  To dump only at the current time regardless of the setting of
      dump limit, use 0 for dump period.
\vspace{.2in}

\noindent
      The dump limit is the maximum number of plots which are to be generated.
      This feature is used in conjunction with dump period to create a file
      containing several plots spaced an equal number of timesteps apart.
\vspace{.2in}

\noindent
      When all desired options have been set, move the highlight bar to OK and
      press \fbox{Enter}.  To exit without plotting to the file and restoring
      old settings, move the highlight bar to CANCEL and press \fbox{Enter}.
\end{subsubsection}

\begin{subsubsection}
{\bf Dot Matrix}

      This option generates a screen dump to the printer at the resolution of
      the screen.  The screen rotated 90 degrees to best fit on a sheet of
      paper.  Note that the resolution resulting on the printer in this case is
      limited to that of the screen.  Dumping to a dot matrix printer may take
      up to one minute depending on the speed of the printer.
\end{subsubsection}

\begin{subsubsection}
{\bf Dump}

      This option is NOT implemented in this version.
\end{subsubsection}
\end{subsection}

\begin{subsection}
{\bf Quit}

   The quit option ends the simulation, exiting back to the DOS prompt.  To
   avoid accidentally selecting quit and loosing the data of a lengthy
   simulation, the user is prompted to ensure that an exit is desired.
   Responding \fbox{OK} will exit, \fbox{CANCEL} will resume the simulation.
\end{subsection}

\begin{subsection}
{\bf Shortcut Keys}

   The menu selections on all menus are chosen either by moving the highlight
   bar to the desired item and pressing \fbox{Enter}, or pressing the
   capitalized letter in the menu item.  For example, to select 
   {\bf stEp}, the user
   could press \fbox{e} or move the highlight bar to {\bf stEp} and 
   press \fbox{Enter}
   .  Note that any unrecognized keystroke will generate a highlight bar if one
   did not exist on the menu, or will be ignored if a highlight bar is
   currently displayed.  The menu manager is insensitive to case; pressing
   \fbox{E} in the previous example is equivalent to \fbox{e}.
\vspace{.2in}

\noindent
   Items in the {\bf Diagnostics} list are selected by pressing the letter
   corresponding to the first character in the name. In the event of
   conflicting first characters, the first matching item is selected.
\end{subsection}
\end{section}

%%%%%%%%%%%%%%%%%%%%%%%%%%%%%%%%%%%%%%%%%%%%%%%%%%%%%

\newpage
\begin{section}
{\bf X-WINDOWS PROGRAM OPERATION}

The X-Windows version of ES1, XES1, is operated in the same manner as discussed
in the previous section, with the exceptions noted below.

\begin{subsection}
{\bf GUI Support}

   XES1 fully supports a mouse for selection of items, buttons etc.  Moving,
   resizing, and iconifying of windows is supported indirectly via the X window
   manager (Motif, Open Look, etc.).  Keystrokes are not supported for these
   actions, so a mouse is required.  The move, resize, and iconifying buttons
   and operations are governed by the window manager; consult the window
   manager manual or guru for details of these procedures.
\end {subsection}

\begin{subsection}
{\bf Main Menu}

   The buttons on the main menu can be selected using the mouse.  The functions
   available include RUN, STOP, STEP, SAVE, and QUIT, which all perform the
   same function described previously in Section 3.  Note that the SAVE
   function is equivalent to the DUMP function in the MS-DOS version which is
   also NOT implemented in this version.
\end{subsection}

\begin{subsection}
{\bf Diagnostic Window Buttons}

   Every diagnostic window in XES1 contains four buttons: Rescale, Trace,
   Print, and Crosshair.

\begin{subsubsection}
{\bf Rescale}

      The rescale button pauses the simulation and opens a dialog box
      containing editable fields for the minimum and maximum labels on the x
      and y axes.  In addition, the dialog box contains buttons for automatic
      rescaling of the x and y axis.  These buttons toggle autorescaling of the
      respective axis on and off.  When all axes are scaled as desired, select
      OK to accept the changes or CANCEL to return to the previous status.
      Note that while rescaling the simulation is paused.
\end{subsubsection}

\begin{subsubsection}
{\bf Trace}

      The trace button turns toggles the plot tracing feature on and off.  The
      previous plots are accumulated, generating a series of lines or dots as
      described above.
\end{subsubsection}

\begin{subsubsection}
{\bf Print}

      The Print button generates a PostScript plot file of the current window.
      Pressing the button opens a dialog box containing the file name for the
      plot and a plot title.  Selecting OK generates the plot, CANCEL returns
      to the simulation.  Note that the simulation is paused while the dialog
      box is open.
\end{subsubsection}

\begin{subsubsection}
{\bf Crosshair}

      The crosshair button activates the crosshair pointer and opens a dialog
      box displaying the coordinates of the pointer.  To display the
      coordinates of a point move the crosshair pointer to the desired location
      and click.  The simulation is paused until the crosshair is deactivated
      by selecting the Crosshair button again.
\end{subsubsection}
\end{subsection}

\begin{subsection}
{\bf Diagnostics}

   In XES1 diagnostics there is no diagnostic menu list.  Instead, all
   diagnostics appear as icons at the bottom of the display or in the
   designated icon area depending on the window manager.  To open a diagnostic,
   simply click on its icon.  In addition, some window managers will display a
   list of the available diagnostics with all other open windows in a window
   list.  Note that this is not a feature of XES1, but rather a feature of the
   X-Windows Manager used on the system.
\end{subsection}
\end{section}

\newpage
\begin{section}
{\bf CASE STUDIES}

ES1 obtains its versatility through the use of input files.  The input file
contains the parameters for the simulation, specifying number of each species,
number of grids, etc.  This section describes the contents, use, and
modification of input files for ES1.

\begin{subsection}
{\bf Input File Parameters}

   The codes use input files to describe the simulation with global parameters,
   as well as the parameters describing each species of particles.  The
   parameters are also described in B\&L (Sect. 3-3).

\begin{subsubsection}
{\bf Global Parameters}
\begin{tabbing}
      mmaxaaaaa\=  The number of particle species to simulate (0= no species present, 1= one \= \kill
      nsp       \>  The number of particle species to simulate (0= no species present, 1= one \> \\
		\>	species in the whole system, etc.).  If
			modifying an input file that has, say, \> \\
		\>	2 species, to add more species,
			just copy one of the blocks of parameters  \> \\
		\>	corresponding to species 1
			or 2, and change the parameters to the desired\> \\
		\>	values.  Note that each species added seeks another 100 k Byte of memory. \> \\
										\\
      $l$         \>    The length of the system. \> \\
							\\
      dt         \>   The time step. \> \\
						\\
      nt        \>     The total number of steps to be run. \\
								\\
      mmax    \>      Maximum number of electrostatic energy modes to be view.
                    Note that \> \\
	      \> this parameter is only for diagnostic purposes
                    and does not enter the \> \\
	       \> calculations. \> \\
								\\
      $l/a$     \>      See Section {\bf 1.4.7} of this manual. \> \\
								\\
      ng     \>        The total number of grid points (power of 2). \> \\
									\\
      iw     \>       Weighting to be used: \> \\
             \>       0  for zero order(NGP) \> \\
  	     \>       1  for first order (CIC, PIC) \> \\
              \>       2  for second order (quadratic spline)  \> \\
              \>       3  for third order (cubic spline) \> \\
								\\
      ec     \>	     Momentum conserving/Energy conserving flag  \> \\
	     \>      0  Momentum conserving scheme (recommended)  \> \\
	     \>      1  Energy conserving scheme             \> \\
	     \>      The momentum conserving scheme uses the same weighting for particles and forces.  \> \\
	     \>      The energy conserving scheme uses a weighting of one lower for forces than is used for particles.  \> \\
								\\
      epsi     \>     1/$\epsilon_{0}$ (usually 1). \> \\
								\\
      a1      \>      Compensation factor (a1 = 0 means no
compensation). \> \\
							\\ 
      a2       \>     Smoothing factor (a2 = 0 means no smoothing). \> \\
										\\
      E0      \>       Magnitude of an applied electric field. \\
								\\
      w0     \>       Frequency of the applied electric field.\> \\
								 \\
      accum   \>      Velocity diagnostics parameter.  0 turns them off  \> \\
	      \>      Other positive integers determine the number of timesteps to accumulate the diagnostics.  \> \\
\end{tabbing}

\end{subsubsection}

\begin{subsubsection}
{\bf Species Parameters}

      One set for each species should be specified.
\begin{tabbing}
      thetaxaaaaa\=Number of sub groups to be given the same velocity
                    distribution, usually \= \kill
      n           \>  Number of particles.\> \\
					\\
      nv2       \>    Exponent of quiet start distribution
		    $f(v) \propto(v/v_{t2})^{nv2}{\rm exp}(-v^{2}/2v_{t2}^{2})$
                    , usually \> \\
		\> zero. \> \\
								\\
      nlg       \>     Number of sub groups to be given the same velocity
                    distribution, usually \> \\ 
		\> one or ng. \> \\
							\\
      mode     \>      Number of mode to be given an initial perturbation in $x$,
                    $v_{x}$. \> \\
								\\
      wp      \>      $\omega_{p}$ (positive). \> \\
								\\
      wc      \>      $\omega_{c}$ (signed). \> \\
								\\
      qm      \>       $q / m$ (signed). \> \\
								\\
      vt1     \>      Provides Gaussian velocity distribution of thermal velocity
                    $v_{t1}$ centered \> \\
		\> on $v_{x}$ = $v_{0}$, $v_{y}$ = 0, 
		    using random number
                    routine; maximum velocity is \> \\ 
		\>  $6v_{t1}$. \> \\
								\\
      vt2       \>    Provides Gaussian (or other) velocity distribution of
                    thermal velocity $v_{t2}$ \> \\
		\>  using inverse 
		    distribution functions,
                    giving ordered velocities ("quiet \> \\
		\> start"). \> \\
								\\
      v0        \>     Drift velocity in $x$ direction (signed). \> \\
								\\
      x1        \>     Magnitude of perturbation in $x$, generally less than half
                    the uniform \> \\
		\> particle spacing, $n/l$; used as
                    $x$1 $\cos ({\rm 2 \pi} x mode / l + \theta_{x})$ \> \\
								\\
      v1        \>     Magnitude of perturbation in $v$; used as
		    $v$1 $\sin ({\rm 2 \pi} x mode / l + \theta_{v})$ \> \\
								\\
      thetax    \>    $\theta_{x}$ \> \\
								\\
      thetav    \>     $\theta_{v}$ \> \\
		\\
      nbins     \>    The number of bins to use when accumulating the velocity \> \\
		\>    distrubution diagnostics.  100 works well.  \\
								\\
      vlower    \>    The lower limit of the velocity diagnostic for this species. \> \\
								\\
      vupper    \>    The upper limit of the velocity diagnostic for this species. \> \\
								\\
\end{tabbing}

\end{subsubsection}
\end{subsection}

\begin{subsection}
{\bf Input File Library}

   Clearly the number of data sets is virtually unlimited.  ES1 is accompanied
   by a number of prepared simulations, many corresponding to the projects
   discussed in Birdsall and Langdon.  The user is encouraged to make a working
   copy of the data sets to edit using any ASCII word processor or editor.
   Note that the relative position of the numeric data is important for useful
   results, but the number of spaces separating each number is unimportant as
   long as the numbers remain on the same line.  Comments may be added on the
   lines containing the descriptive text (so long as it remains a single line)
   and at the end of the input file an unlimited number of text lines are
   supported.

\begin{subsubsection}
{\bf 2STREAM.INP}

      The two stream instability is far more instructive on the PC since we can
      observe the growth and streaming take place rather than viewing
      intermittent phase plots (a la Cray output).  In addition, we use a
      different color for each species so the species can be tracked as they
      warm up.  This is a good input file to try the NoErase option on to find
      the trapping regions.  Can observe growth of the instability (and
      eventual saturation) using energy histories.  See Birdsall and Langdon,
      Secs. 5-8 and 5-9.
\end{subsubsection}

\begin{subsubsection}
{\bf 3STREAM.INP}

      Similar to 2STREAM, except we add a cold non-drifting species.  Also try
      the NoErase option to observe the trapping regions.
\end{subsubsection}

\begin{subsubsection}
{\bf 4E4E.INP}

      Similar to a two stream instability, we can look at just 4 particles to
      observe streaming, trapping, etc.  Use the NoErase option to view
      trapping regions.
\end{subsubsection}

\begin{subsubsection}
{\bf 4E4P.INP}

      We try to form atoms with 8 particles, and observe that when an atom-pair
      is beginning to stabilize, another particle tends to interfere and break
      the pair up.
\end{subsubsection}

\begin{subsubsection}
{\bf 4STREAM.INP}

      Similar to 2STREAM, we use 4 streams to observe a double two stream
      instability.  Also try NoErase to observe the trapping regions.
\end{subsubsection}

\begin{subsubsection}
{\bf BEAMCYC.INP}

      Beam cyclotron instability; can observe growth using energy histories.
      See Birdsall and Langdon Sec. 5-14.
\end{subsubsection}

\begin{subsubsection}
{\bf BEAMPLAS.INP}

      Beam plasma instability; can observe growth using energy histories.  See
      Birdsall and Langdon Sec. 5-12.
\end{subsubsection}

\begin{subsubsection}
{\bf COLD17.INP}

      Similar to COLDPLAS, except we perturb in mode 17 with NG=32 to observe
      aliasing back to mode 15 in density.  Also try perturbing in mode 32 to
      get equivalent of a mode 0 perturbation - nothing happens.  See Birdsall
      and Langdon page 92.
\end{subsubsection}

\begin{subsubsection}
{\bf COLDBEAM.INP}

      Cold beam numerical instability; can observe aliasing effects using phase
      space plot and any grid diagnostic.  Can observe growth of the
      instability using energy histories.  See Birdsall and Langdon Sec. 8-12.
\end{subsubsection}

\begin{subsubsection}
{\bf COLDPLAS.INP}

      Cold plasma oscillations similar to TWOOSC except we add many particles
      (this is a mode 1 perturbation).  See Birdsall and Langdon Sec. 5-4.
\end{subsubsection}

\begin{subsubsection}
{\bf EE.INP}

      Similar to 4E4E, we can observe the two stream instability with 8
      particles.
\end{subsubsection}

\begin{subsubsection}
{\bf EP.INP}

      Loading an electron and positron of equal masses and opposite drift, we
      form a 'atom' (this is not clearly observable on batch machines with
      intermittent plots).  Try the NoErase option (Trace in the X-Windows
      version) to observe the trajectory of the particles in time.
\end{subsubsection}

\begin{subsubsection}
{\bf EPMASS.INP}

      Similar to EP except the positive particle is 1000 times more massive
      than the negative particle.  We end up with a translating atom with the
      electron vibrating about the proton.
\end{subsubsection}

\begin{subsubsection}
{\bf HYBRID.INP}

      Hybrid oscillations, similar to those described in Sec. 5-5 in Birdsall
      and Langdon.  The oscillations are similar to cold plasma oscillations,
      except the plasma is in a magnetic field which changes the frequency of
      oscillation to the hybrid frequency.  See Birdsall and Langdon Sec. 5-5.
\end{subsubsection}

\begin{subsubsection}
{\bf LANDAU.INP}

      Landau damping using only 2048 particles.  This one may be a little slow
      on a 4 MHz PC; it takes about 1.5 seconds per timestep on an 8 MHz IBM
      AT.  See Birdsall and Langdon Sec. 5-15.
\end{subsubsection}

\begin{subsubsection}
{\bf LANDAUP.INP}

  Using LANDAU.INP with the excitation of mode 1 at $k \lambda_D = 0.4$,
it is found that $\omega_{real} = 1.2 \omega_p$, which makes the wave 
phase velocity $\omega_{real}/k = 1.2$.  In order to observe the phase 
space in the frame of the wave, shift the Maxwellian by $v_{phase}$ by
using $v_0 = -1.2.$  Now the $v_x=0$ axis is the wave frame.  At $t=0$
expand the $v_x$ axes to +- 0.2, turn on the trace, and run the simulation.
Observe the very few particles near $v=0$ escape the wave trap and become
passing particles, as the wave damps.  Next, reduce the initial excitation
until the Landau damping rate reduces to that in texts, and observe the
phase space change from trapping to phase mixing.  Next increase the initial
excitation until trapping dominates.

\end{subsubsection}

\begin{subsubsection}
{\bf TWOOSC.INP}

      Two particle cold oscillations.  This is more instructive to run on the
      PC than on the Cray since one can observe particles bouncing in simple
      harmonic motion.  A good demonstration of ES1 movers is to increase V1 so
      the particles pass through each other without elastic collision to
      convince yourself that ES1 avoids singularity in fields.
\end{subsubsection}
\end{subsection}

\begin{subsection}
{\bf Format}

   The input file is currently a fixed format ASCII file.  The input files may
   be edited using any ASCII text editor or word processor which does not
   insert formatting characters into the file.  The text lines can contain any
   descriptive comments, etc., and the text may continue for as many lines as
   desired.  The lines containing numbers must also remain on a single line and
   each number can only be separated by whitespace, including any amount of
   spaces and tabs.  The numbers may be in floating point or exponential format
   (decimal point and sign are optional).
\vspace{.1in}

\noindent
   The preferred method of trying new input parameters is to COPY the original
   data set to a new file name and edit the working copy.  This will leave the
   original data set intact for future use and reference.
\vspace{.2in}

\noindent
   The syntax of the COPY command is given in the DOS manual.  For example, to
   edit the Landau Damping input file, first type 
   {\sc COPY LANDAU.INP LANDAU1.INP} to create a copy of the original 
   data.  Now the copy, LANDAU1.INP, can be
   edited using the instructions included with your editor.
\end{subsection}
\end{section}

\newpage
\noindent
\begin{appendix}
\begin{section}
{\bf APPENDIX SYSTEM REQUIREMENTS}
The minimum MSDOS hardware configuration required to run ES1 is as follows:

\begin{tabbing}
     $\circ$aaaaa\=   IBM XT, AT, PS/2, or compatible computer.  (80286 or 80386
          recommended) \= \kill
     $\circ$ \>   IBM XT, AT, PS/2, or compatible computer.  (80286 or 80386
          recommended) \> \\

     $\circ$ \>   DOS version 3.0 and above. \> \\

     $\circ$ \>   512K system memory (640K recommended). \> \\

     $\circ$ \>  IBM-compatible EGA, or VGA color graphics adapter (EGA or VGA
          and color \> \\
	     \>
		monitor required). \> \\

     $\circ$ \>   8087/80x87 numeric coprocessor recommended (provides increase in
          speed of \> \\
	     \> over an order of magnitude). \> 
\end{tabbing}

\noindent
In addition, a 32 bit 80386 or 80486-based processor running at a high clock
speed is recommended for increased performance.
\vspace{.2in}

\noindent
The X-Windows version requires X11 libraries (X11R4 or any superset of X11 such
as Motif), a C compiler, and an X display or X-terminal.
\end{section}

\newpage
\noindent
\begin{section}
{\bf APPENDIX TECHNICAL SUPPORT}

Current users interested in new versions should contact us periodically; we
currently are not staffed to notify our entire user base.  Technical questions,
comments, and suggestions on this software and documentation can be sent to:

\noindent
     birdsall@eecs.berkeley.edu
\vspace{0.2in}

\noindent
        Prof. C. K. Birdsall                    			\\
        EECS Dept., Cory Hall                   			\\
        University of California	       				\\
        Berkeley, CA  94720                     			\\
\end{section}
\end{appendix}
\end{document}
